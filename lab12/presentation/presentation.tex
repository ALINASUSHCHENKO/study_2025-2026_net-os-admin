\input{preamble.tex}

\title{Лабораторная работа № 12. \\Синхронизация времени}
\author{Сущенко Алина \\ НПИбд-01-23}
\institute{Российский университет дружбы народов имени Патриса Лумумбы}
\date{2025}

\begin{document}

\frame{\titlepage}

\begin{frame}
  \frametitle{Цель работы}
  \begin{itemize}
  \item Получение навыков по управлению системным временем и настройке синхронизации времени.
  \end{itemize}
\end{frame}

\begin{frame}
  \frametitle{Настройка параметров времени}
  \centering
  \includegraphics[width=\textwidth]{../images/1.png}
  \captionof{figure}{Информация о дате и времени на сервере (timedatectl).}
\end{frame}

\begin{frame}
  \frametitle{Настройка параметров времени}
  \centering
  \includegraphics[width=\textwidth]{../images/2.png}
  \captionof{figure}{Информация о дате и времени на клиенте (timedatectl).}
\end{frame}

\begin{frame}
  \frametitle{Настройка параметров времени}
  \centering
  \includegraphics[width=\textwidth]{../images/4.png}
  \captionof{figure}{Вывод команды \texttt{date} на сервере.}
\end{frame}

\begin{frame}
  \frametitle{Настройка параметров времени}
  \centering
  \includegraphics[width=\textwidth]{../images/3.png}
  \captionof{figure}{Вывод команды \texttt{date} на клиенте.}
\end{frame}

\begin{frame}[fragile]
\frametitle{Управление синхронизацией времени}
  \begin{minted}{bash}
    dnf -y install chrony
  \end{minted}
\end{frame}

\begin{frame}
\frametitle{Управление синхронизацией времени}
    \centering
    \includegraphics[width=\textwidth]{../images/5.png}
    \captionof{figure}{Проверка источников времени на клиенте.}
\end{frame}

\begin{frame}[fragile]
\frametitle{Управление синхронизацией времени}
На сервере открыли на редактирование файл \texttt{/etc/chrony.conf} и добавили строку:
  \begin{minted}{bash}
    allow 192.168.0.0/16
  \end{minted}
    \centering
    \includegraphics[width=\textwidth]{../images/6.png}
    \captionof{figure}{Строчка в конфиурационном файле.}
\end{frame}

\begin{frame}
\frametitle{Управление синхронизацией времени}
\centering
    \includegraphics[width=\textwidth]{../images/7.png}
    \captionof{figure}{Перезапуск \texttt{chronyd} и настройка межсетевого экрана.}
\end{frame}

\begin{frame}
\frametitle{Управление синхронизацией времени}
    \centering
    \includegraphics[width=\textwidth]{../images/8.png}
    \captionof{figure}{Изменение файла конфигурации \texttt{chrony}}
\end{frame}

\begin{frame}
\frametitle{Управление синхронизацией времени}
    \centering
    \includegraphics[width=\textwidth]{../images/10.png}
    \captionof{figure}{Просмотр источников времени на сервере.}
\end{frame}

\begin{frame}
\frametitle{Внесение изменений в настройки внутреннего окружения виртуальных машин}
    \centering
    \includegraphics[width=\textwidth]{../images/9.png}
    \captionof{figure}{Проверка добавленного источника времени на клиенте.}
\end{frame}

\begin{frame}[fragile]
\frametitle{Внесение изменений в настройки внутреннего окружения виртуальных машин}
  \begin{minted}{bash}
    #!/bin/bash
    echo "Provisioning script $0"
    echo "Install needed packages"
    dnf -y install chrony
    echo "Copy configuration files"
    cp -R /vagrant/provision/server/ntp/etc/* /etc
    restorecon -vR /etc
    echo "Configure firewall"
    firewall-cmd --add-service=ntp
    firewall-cmd --add-service=ntp --permanent
    echo "Restart chronyd service"
    systemctl restart chronyd
  \end{minted}

\end{frame}

\begin{frame}
\frametitle{Внесение изменений в настройки внутреннего окружения виртуальных машин}
    \centering
    \includegraphics[width=\textwidth]{../images/11.png}
    \captionof{figure}{Настройка внутреннего окружения виртуальной машины клиента.}
\end{frame}

\begin{frame}[fragile]
\frametitle{Внесение изменений в настройки внутреннего окружения виртуальных машин}
  \begin{minted}{bash}
    #!/bin/bash
    echo "Provisioning script $0"
    echo "Copy configuration files"
    cp -R /vagrant/provision/client/ntp/etc/* /etc
    restorecon -vR /etc
    echo "Restart chronyd service"
    systemctl restart chronyd
  \end{minted}
\end{frame}

\begin{frame}[fragile]
\frametitle{Внесение изменений в настройки внутреннего окружения виртуальных машин}
  \begin{minted}{bash}
    server.vm.provision "server ntp",
    type: "shell",
    preserve_order: true,
    path: "provision/server/ntp.sh"
    client.vm.provision "client ntp",
    type: "shell",
    preserve_order: true,
    path: "provision/client/ntp.sh"
  \end{minted}
\end{frame}

\begin{frame}
\frametitle{Контрольные вопросы}
\begin{enumerate}
    \item Почему важна точная синхронизация времени для служб баз данных?
    \item Почему служба проверки подлинности Kerberos сильно зависит от правильной синхронизации времени?
    \item Какая служба используется по умолчанию для синхронизации времени на RHEL 7?
    \item Какова страта по умолчанию для локальных часов?
    \item Какой порт брандмауэра должен быть открыт, если вы настраиваете свой сервер как одноранговый узел NTP?
    \item Какую строку вам нужно включить в конфигурационный файл chrony, если вы хотите быть сервером времени, даже если внешние серверы NTP недоступны?
    \item Какую страту имеет хост, если нет текущей синхронизации времени NTP?
    \item Какую команду вы бы использовали на сервере с chrony, чтобы узнать, с какими серверами он синхронизируется?
    \item Как вы можете получить подробную статистику текущих настроек времени для процесса chrony вашего сервера?
\end{enumerate}
\end{frame}

\begin{frame}
\frametitle{Ответы на контрольные вопросы}
\begin{enumerate}
    \item \textbf{Точная синхронизация времени важна для служб баз данных} потому что обеспечивает корректность временных меток транзакций, согласованность репликации данных между серверами и предотвращает конфликты при распределенных операциях.
\end{enumerate}
\end{frame}

\begin{frame}
\frametitle{Ответы на контрольные вопросы}
\begin{enumerate}
    \setcounter{enumi}{1}
    \item \textbf{Kerberos сильно зависит от синхронизации времени} потому что использует временные метки в билетах аутентификации для предотвращения replay-атак. Расхождение во времени более 5 минут обычно приводит к отказу в аутентификации.
\end{enumerate}
\end{frame}

\begin{frame}
\frametitle{Ответы на контрольные вопросы}
\begin{enumerate}
    \setcounter{enumi}{2}
    \item \textbf{В RHEL 7 по умолчанию используется служба chronyd} для синхронизации времени.
\end{enumerate}
\end{frame}

\begin{frame}
\frametitle{Ответы на контрольные вопросы}
\begin{enumerate}
    \setcounter{enumi}{3}
    \item \textbf{Страта по умолчанию для локальных часов} равна 10, что указывает на их низкую точность по сравнению с внешними NTP-серверами.
\end{enumerate}
\end{frame}

\begin{frame}
\frametitle{Ответы на контрольные вопросы}
\begin{enumerate}
    \setcounter{enumi}{4}
    \item \textbf{Для NTP-сервера нужно открыть порт 123/UDP}, который используется для протокола NTP.
\end{enumerate}
\end{frame}

\begin{frame}
\frametitle{Ответы на контрольные вопросы}
\begin{enumerate}
    \setcounter{enumi}{5}
    \item \textbf{Нужно добавить строку:} \texttt{local stratum 10} в файл \texttt{/etc/chrony.conf}, чтобы сервер мог работать как источник времени даже при отсутствии внешних серверов.
\end{enumerate}
\end{frame}

\begin{frame}
\frametitle{Ответы на контрольные вопросы}
\begin{enumerate}
    \setcounter{enumi}{6}
    \item \textbf{При отсутствии синхронизации NTP хост имеет страту 16}, что означает несинхронизированное состояние.
\end{enumerate}
\end{frame}

\begin{frame}
\frametitle{Ответы на контрольные вопросы}
\begin{enumerate}
    \setcounter{enumi}{7}
    \item \textbf{Команда для просмотра серверов синхронизации:} \texttt{chronyc sources} или \texttt{chronyc sources -v}
\end{enumerate}
\end{frame}

\begin{frame}
\frametitle{Ответы на контрольные вопросы}
\begin{enumerate}
    \setcounter{enumi}{8}
    \item \textbf{Для получения подробной статистики используется команда:} \texttt{chronyc tracking} - показывает детальную информацию о текущих настройках времени и статистике синхронизации.
\end{enumerate}
\end{frame}

\begin{frame}
\frametitle{Выводы}
\begin{itemize}
    \item В результате выполнения лабораторной работы получили навыки по управлению системным временем и настройке синхронизации времени.
    \item Освоили работу с chrony для настройки NTP-сервера и клиента.
    \item Изучили основные команды для мониторинга и управления синхронизацией времени.
\end{itemize}
\end{frame}
\end{document}
