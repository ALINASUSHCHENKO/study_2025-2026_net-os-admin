\input{preamble.tex}

\title{Лабораторная работа № 10. \\ Расширенные настройки SMTP-сервера}
\author{Сущенко Алина \\ НПИбд-01-23}

\institute{Российский университет дружбы народов имени Патриса Лумумбы}
\date{2025}

\begin{document}

\frame{\titlepage}

\begin{frame}
\frametitle{Цель работы}
\begin{itemize}
    \item Приобретение практических навыков по конфигурированию SMTP-сервера в части настройки аутентификации.
\end{itemize}
\end{frame}

\begin{frame}[fragile]
\frametitle{Настройка LMTP в Dovecot}
\begin{enumerate}
\item На виртуальной машине \texttt{server} вошли под пользователем и открыли терминал. Перешли в режим суперпользователя:
\begin{minted}{bash}
sudo -i
\end{minted}
\item В дополнительном терминале запустили мониторинг работы почтовой службы:
\begin{minted}{bash}
sudo -i
tail -f /var/log/maillog
\end{minted}
\end{enumerate}
\end{frame}

\begin{frame}[fragile]
\frametitle{Настройка LMTP в Dovecot}
\begin{enumerate}
\setcounter{enumi}{2}
\item Добавили в список протоколов, с которыми может работать Dovecot, протокол LMTP:
\begin{minted}{bash}
protocols = imap pop3 lmtp
\end{minted}
\end{enumerate}
\centering
\includegraphics[width=0.8\textwidth]{../images/1.png}
\captionof{figure}{Обновление списка разрешенных протоколов в Dovecot.}
\end{frame}

\begin{frame}[fragile]
\frametitle{Настройка LMTP в Dovecot}
\begin{enumerate}
\setcounter{enumi}{3}
\item Настроили в Dovecot сервис \texttt{lmtp} для связи с Postfix:
\begin{minted}{bash}
service lmtp {
  unix_listener /var/spool/postfix/private/dovecot-lmtp {
    group = postfix
    user = postfix
    mode = 0600
  }
}
\end{minted}
\end{enumerate}
\centering
\includegraphics[width=0.8\textwidth]{../images/2.png}
\captionof{figure}{Настройки сервера \texttt{lmtp}.}
\end{frame}

\begin{frame}[fragile]
\frametitle{Настройка LMTP в Dovecot}
\begin{enumerate}
\setcounter{enumi}{4}
\item Переопределили в Postfix передачу сообщений через unix-сокет:
\begin{minted}{bash}
postconf -e 'mailbox_transport = lmtp:unix:private/dovecot-lmtp'
\end{minted}
\item Задали формат имени пользователя для аутентификации:
\begin{minted}{bash}
auth_username_format = %Ln
\end{minted}
\end{enumerate}
\centering
\includegraphics[width=0.8\textwidth]{../images/3.png}
\captionof{figure}{Задание формата имени пользователя для аутентификации.}
\end{frame}

\begin{frame}[fragile]
\frametitle{Настройка LMTP в Dovecot}
\begin{enumerate}
\setcounter{enumi}{6}
\item Перезапустили Postfix и Dovecot:
\begin{minted}{bash}
systemctl restart postfix
systemctl restart dovecot
\end{minted}
\end{enumerate}
\centering
\includegraphics[width=0.8\textwidth]{../images/4.png}
\captionof{figure}{Перезагрузка Postfix и Dovecot.}
\end{frame}

\begin{frame}[fragile]
\frametitle{Настройка LMTP в Dovecot}
\begin{enumerate}
\setcounter{enumi}{7}
\item Отправили тестовое письмо:
\begin{minted}{bash}
echo .| mail -s "LMTP test" ansusenko@ansusenko.net
\end{minted}
\item Просмотрели почтовый ящик пользователя:
\begin{minted}{bash}
MAIL=~/Maildir/ mail
\end{minted}
\end{enumerate}
\centering
\includegraphics[width=0.8\textwidth]{../images/5.png}
\captionof{figure}{Просмотр доставленного письма через утилиту \texttt{mail}.}
\end{frame}

\begin{frame}
\frametitle{Настройка LMTP в Dovecot}
\centering
\includegraphics[width=0.8\textwidth]{../images/6.png}
\captionof{figure}{Просмотр логов при отправке письма.}
\end{frame}

\begin{frame}[fragile]
\frametitle{Настройка SMTP-аутентификации}
\begin{enumerate}
\item Определили службу аутентификации пользователей:
\begin{minted}{bash}
service auth {
  unix_listener /var/spool/postfix/private/auth {
    group = postfix
    user = postfix
    mode = 0660
  }
  unix_listener auth-userdb {
    mode = 0600
    user = dovecot
  }
}
\end{minted}
\end{enumerate}
\centering
\includegraphics[width=0.8\textwidth]{../images/7.png}
\captionof{figure}{Настройка службы аутентификации.}
\end{frame}

\begin{frame}[fragile]
\frametitle{Настройка SMTP-аутентификации}
\begin{enumerate}
\setcounter{enumi}{1}
\item Задали тип аутентификации SASL для Postfix:
\begin{minted}{bash}
postconf -e 'smtpd_sasl_type = dovecot'
postconf -e 'smtpd_sasl_path = private/auth'
\end{minted}
\end{enumerate}
\centering
\includegraphics[width=0.8\textwidth]{../images/8.png}
\captionof{figure}{Настройка типа аутентификации SASL для \texttt{smtpd}.}
\end{frame}

\begin{frame}[fragile]
\frametitle{Настройка SMTP-аутентификации}
\begin{enumerate}
\setcounter{enumi}{2}
\item Настроили Postfix для защиты от спамрассылок:
\begin{minted}{bash}
postconf -e 'smtpd_recipient_restrictions = reject_unknown_recipient_domain, permit_mynetworks, reject_non_fqdn_recipient, reject_unauth_destination, reject_unverified_recipient, permit'
\end{minted}
\end{enumerate}
\centering
\includegraphics[width=0.8\textwidth]{../images/9.png}
\captionof{figure}{Настройка Postfix для запрета спамрассылок.}
\end{frame}

\begin{frame}[fragile]
\frametitle{Настройка SMTP-аутентификации}
\begin{enumerate}
\setcounter{enumi}{3}
\item Ограничили приём почты только локальным адресом:
\begin{minted}{bash}
postconf -e 'mynetworks = 127.0.0.0/8'
\end{minted}
\end{enumerate}
\centering
\includegraphics[width=0.8\textwidth]{../images/10.png}
\captionof{figure}{Ограничение приема почты только локальным адресом.}
\end{frame}

\begin{frame}[fragile]
\frametitle{Настройка SMTP-аутентификации}
\begin{enumerate}
\setcounter{enumi}{4}
\item Настроили SMTP-сервер с возможностью аутентификации:
\begin{minted}{bash}
smtp inet n - n - - smtpd
-o smtpd_sasl_auth_enable=yes
-o smtpd_recipient_restrictions=reject_non_fqdn_recipient, reject_unknown_recipient_domain, permit_sasl_authenticated, reject
\end{minted}
\end{enumerate}
\end{frame}

\begin{frame}[fragile]
\frametitle{Настройка SMTP-аутентификации}
\begin{enumerate}
\setcounter{enumi}{5}
\item Перезапустили Postfix и Dovecot:
\begin{minted}{bash}
systemctl restart postfix
systemctl restart dovecot
\end{minted}
\end{enumerate}
\centering
\includegraphics[width=0.8\textwidth]{../images/11.png}
\captionof{figure}{Перезагрузка Postfix и Dovecot.}
\end{frame}

\begin{frame}[fragile]
\frametitle{Настройка SMTP-аутентификации}
\begin{enumerate}
\setcounter{enumi}{6}
\item Установили telnet на клиенте:
\begin{minted}{bash}
sudo -i
dnf -y install telnet
\end{minted}
\item Протестировали соединение:
\begin{minted}{bash}
EHLO test
AUTH PLAIN <строка для аутентификации>
\end{minted}
\end{enumerate}
\centering
\includegraphics[width=0.8\textwidth]{../images/12.png}
\captionof{figure}{Подключение к SMTP-серверу через \texttt{telnet}.}
\end{frame}

\begin{frame}[fragile]
\frametitle{Настройка SMTP over TLS}
\begin{minted}{bash}
На Alpine выполним подключение по порту, отправим себе письмо и проверим вывод
\end{minted}
\centering
\includegraphics[width=0.8\textwidth]{../images/13.png}
\captionof{figure}{Отправка сообщения \texttt{telnet}.}
\end{frame}

\begin{frame}[fragile]
\frametitle{Внесение изменений в настройки внутреннего окружения}
\begin{enumerate}
\item Скопировали конфигурационные файлы:
\begin{minted}{bash}
cd /vagrant/provision/server
cp -R /etc/dovecot/dovecot.conf /vagrant/provision/server/mail/etc/dovecot/
cp -R /etc/dovecot/conf.d/10-master.conf /vagrant/provision/server/mail/etc/dovecot/conf.d/
cp -R /etc/dovecot/conf.d/10-auth.conf /vagrant/provision/server/mail/etc/dovecot/conf.d/
mkdir -p /vagrant/provision/server/mail/etc/postfix/
cp -R /etc/postfix/master.cf /vagrant/provision/server/mail/etc/postfix/
\end{minted}
\end{enumerate}
\end{frame}

\begin{frame}[fragile]
\frametitle{Выводы}
\begin{itemize}
    \item В результате выполнения лабораторной работы приобрели практические навыки по конфигурированию SMTP-сервера в части настройки аутентификации.
    \item Освоили настройку LMTP в Dovecot для интеграции с Postfix.
    \item Настроили SMTP-аутентификацию с использованием SASL.
    \item Реализовали защиту почтового сервера от использования в качестве открытого релея.
    \item Настроили работу SMTP over TLS для безопасной передачи почты.
\end{itemize}
\end{frame}

\begin{frame}[fragile]
\frametitle{Контрольные вопросы}
\begin{block}{1. Приведите пример задания формата аутентификации пользователя в Dovecot в форме логина с указанием домена.}
\begin{itemize}
\item Для указания формата логина с доменом используется параметр:
\begin{minted}{bash}
auth_username_format = %Lu
\end{minted}
\item Альтернативный вариант - указание полного формата:
\begin{minted}{bash}
auth_username_format = %n
\end{minted}
\end{itemize}
\end{block}
\end{frame}

\begin{frame}
\frametitle{Контрольные вопросы}
\begin{block}{2. Какие функции выполняет почтовый Relay-сервер?}
\begin{itemize}
\item \textbf{Пересылка почты} - передача сообщений между разными почтовыми серверами
\item \textbf{Кэширование} - временное хранение сообщений при проблемах с доставкой
\item \textbf{Балансировка нагрузки} - распределение почтового трафика
\item \textbf{Фильтрация} - проверка почты на спам и вирусы
\item \textbf{Анонимизация} - скрытие реального источника отправки почты
\end{itemize}
\end{block}
\end{frame}

\begin{frame}
\frametitle{Контрольные вопросы}
\begin{block}{3. Какие угрозы безопасности могут возникнуть в случае настройки почтового сервера как Relay-сервера?}
\begin{itemize}
\item \textbf{Спам-рассылки} - использование сервера для массовой рассылки спама
\item \textbf{Черные списки} - IP-адрес сервера может быть внесен в DNSBL
\item \textbf{Перегрузка ресурсов} - большой объем несанкционированной почты
\item \textbf{Компрометация репутации} - ухудшение репутации домена и IP-адреса
\item \textbf{Юридические риски} - ответственность за рассылку спама
\end{itemize}
\end{block}
\end{frame}

\begin{frame}
\frametitle{Меры защиты от угроз Relay-сервера}
\begin{itemize}
\item \textbf{Аутентификация} - требовать аутентификацию для отправки почты
\item \textbf{Ограничение сетей} - настройка \texttt{mynetworks} только для доверенных сетей
\item \textbf{Проверка получателей} - использование \texttt{smtpd\_recipient\_restrictions}
\item \textbf{Шифрование} - обязательное использование TLS для передачи почты
\item \textbf{Мониторинг} - регулярный анализ логов почтовой активности
\item \textbf{Ограничение квот} - установка лимитов на объем отправляемой почты
\end{itemize}
\end{frame}

\end{document}
