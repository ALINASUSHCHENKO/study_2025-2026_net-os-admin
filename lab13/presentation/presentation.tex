\input{preamble.tex}

\title{Лабораторная работа № 13. \\ Настройка NFS}
\author{Сущенко Алина \\ НПИбд-01-23}
\institute{Российский университет дружбы народов имени Патриса Лумумбы}
\date{2025}

\begin{document}

\frame{\titlepage}

\begin{frame}
\frametitle{Цель работы}
\begin{itemize}
    \item Приобретение навыков настройки сервера NFS для удалённого доступа к ресурсам.
\end{itemize}
\end{frame}

\begin{frame}
\frametitle{Настройки сервера NFSv4}
    \includegraphics[width=\textwidth]{../images/1.png}
\end{frame}

\begin{frame}
\frametitle{Настройки сервера NFSv4}
    \includegraphics[width=\textwidth]{../images/2.png}
\end{frame}

\begin{frame}
\frametitle{Настройки сервера NFSv4}
    \includegraphics[width=\textwidth]{../images/3.png}
\end{frame}

\begin{frame}
\frametitle{Настройки сервера NFSv4}
    \includegraphics[width=\textwidth]{../images/4.png}
\end{frame}

\begin{frame}
\frametitle{Настройки сервера NFSv4}
    \includegraphics[width=\textwidth]{../images/5.png}
\end{frame}

\begin{frame}
\frametitle{Настройки сервера NFSv4}
    \includegraphics[width=\textwidth]{../images/6.png}
\end{frame}

\begin{frame}
\frametitle{Настройки сервера NFSv4}
    \includegraphics[width=\textwidth]{../images/7.png}
\end{frame}

\begin{frame}
\frametitle{Настройки сервера NFSv4}
    \includegraphics[width=\textwidth]{../images/8.png}
\end{frame}

\begin{frame}
\frametitle{Монтирование NFS на клиенте}
    \includegraphics[width=\textwidth]{../images/9.png}
\end{frame}

\begin{frame}
\frametitle{Монтирование NFS на клиенте}
    \includegraphics[width=\textwidth]{../images/10.png}
\end{frame}

\begin{frame}
\frametitle{Монтирование NFS на клиенте}
    \includegraphics[width=\textwidth]{../images/11.png}
\end{frame}

\begin{frame}
\frametitle{Монтирование NFS на клиенte}
    \includegraphics[width=\textwidth]{../images/12.png}
\end{frame}

\begin{frame}
\frametitle{Монтирование NFS на клиенте}
    \includegraphics[width=\textwidth]{../images/13.png}
\end{frame}

\begin{frame}
\frametitle{Подключение каталогов к дереву NFS}
    \includegraphics[width=\textwidth]{../images/15.png}
\end{frame}

\begin{frame}
\frametitle{Подключение каталогов к дереву NFS}
    \includegraphics[width=\textwidth]{../images/13.png}
\end{frame}

\begin{frame}
\frametitle{Подключение каталогов к дереву NFS}
    \includegraphics[width=\textwidth]{../images/14.png}
\end{frame}

\begin{frame}
\frametitle{Подключение каталогов к дереву NFS}
    \includegraphics[width=\textwidth]{../images/15.png}
\end{frame}

\begin{frame}
\frametitle{Подключение каталогов к дереву NFS}
    \includegraphics[width=\textwidth]{../images/16.png}
\end{frame}

\begin{frame}
\frametitle{Подключение каталогов к дереву NFS}
    \includegraphics[width=\textwidth]{../images/17.png}
\end{frame}

\begin{frame}
\frametitle{Подключение каталогов для работы пользователей}
    \includegraphics[width=\textwidth]{../images/18.png}
\end{frame}

\begin{frame}
\frametitle{Подключение каталогов для работы пользователей}
    \includegraphics[width=\textwidth]{../images/19.png}
\end{frame}

\begin{frame}
\frametitle{Подключение каталогов для работы пользователей}
    \includegraphics[width=\textwidth]{../images/20.png}
\end{frame}

\begin{frame}
\frametitle{Подключение каталогов для работы пользователей}
    \includegraphics[width=\textwidth]{../images/21.png}
\end{frame}

\begin{frame}
\frametitle{Подключение каталогов для работы пользователей}
    \includegraphics[width=\textwidth]{../images/23.png}
\end{frame}

\begin{frame}
\frametitle{Контрольные вопросы 1}
\textbf{1. Как называется файл конфигурации, содержащий общие ресурсы NFS?}

Основной файл конфигурации NFS: \texttt{/etc/exports}

В этом файле указываются:
\begin{itemize}
\item Экспортируемые каталоги
\item Разрешенные клиенты или сети  
\item Права доступа (ro/rw)
\item Дополнительные опции
\end{itemize}
\end{frame}

\begin{frame}
\frametitle{Контрольные вопросы 2}
\textbf{2. Какие порты должны быть открыты в брандмауэре?}

Необходимые службы:
\begin{itemize}
\item \texttt{nfs} - порт 2049
\item \texttt{mountd} - динамический порт
\item \texttt{rpc-bind} - порт 111
\item \texttt{nfs3} - для NFSv3
\end{itemize}

Команды для FirewallD:
\begin{itemize}
\item \texttt{firewall-cmd --add-service=nfs --permanent}
\item \texttt{firewall-cmd --add-service=mountd --permanent}
\item \texttt{firewall-cmd --add-service=rpc-bind --permanent}
\item \texttt{firewall-cmd --reload}
\end{itemize}
\end{frame}

\begin{frame}
\frametitle{Контрольные вопросы 3}
\textbf{3. Опция для автоматического монтирования NFS при перезагрузке?}

Опция: \texttt{\_netdev}

Пример записи в \texttt{/etc/fstab}:
\begin{center}
\texttt{server.ansusenko.net:/srv/nfs /mnt/nfs nfs \_netdev 0 0}
\end{center}

\begin{itemize}
\item \texttt{\_netdev} - ждет инициализации сети
\item \texttt{nfs} - тип файловой системы
\item \texttt{0 0} - параметры для dump/fsck
\end{itemize}
\end{frame}

\begin{frame}
\frametitle{Выводы}
\begin{itemize}
    \item Приобрели навыки настройки сервера NFS
    \item Настроили удаленный доступ к ресурсам
    \item Создали каталоги для веб-сервера и пользователей
\end{itemize}
\end{frame}

\end{document}
