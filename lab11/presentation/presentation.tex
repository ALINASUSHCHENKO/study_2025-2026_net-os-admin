\input{preamble.tex}

\title{Лабораторная работа № 11. \\ Настройка безопасного удалённого доступа по протоколу SSH}
\author{Сущенко Алина \\ НПИбд-01-23}
\institute{Российский университет дружбы народов имени Патриса Лумумбы}
\date{2025}

\begin{document}

\frame{\titlepage}

\begin{frame}
\frametitle{Цель работы}
\begin{itemize}
    \item Приобретение практических навыков по настройке удалённого доступа к серверу с помощью SSH.
\end{itemize}
\end{frame}

\begin{frame}
\frametitle{Запрет удалённого доступа по SSH для пользователя root}
  \centering
  \includegraphics[width=\textwidth]{../images/1.png}
  \captionof{figure}{Задание пароля для пользователя root.}
\end{frame}

\begin{frame}[fragile]
\frametitle{Запрет удалённого доступа по SSH для пользователя root}
Запретили пользователю root подключение к серверу через SSH:
  \begin{minted}{bash}
    PermitRootLogin no
  \end{minted}
  \centering
  \includegraphics[width=\textwidth]{../images/2.png}
  \captionof{figure}{Подключение к серверу через SSH-соединение.}
\end{frame}

\begin{frame}
\frametitle{Ограничение списка пользователей для удалённого доступа по SSH}
  \centering
  \includegraphics[width=\textwidth]{../images/3.png}
  \captionof{figure}{Успешное подключение к серверу пользователем ansusenko.}
\end{frame}

\begin{frame}[fragile]
\frametitle{Ограничение списка пользователей для удалённого доступа по SSH}
Явно указали разрешенных к подключению пользователей:
  \begin{minted}{bash}
    AllowUsers vagrant
  \end{minted}
  \centering
  \includegraphics[width=\textwidth]{../images/4.png}
  \captionof{figure}{Отказ в доступе на сервер пользователю ansusenko.}
\end{frame}

\begin{frame}[fragile]
\frametitle{Ограничение списка пользователей для удалённого доступа по SSH}
Обновили список разрешенных пользователей:
  \begin{minted}{bash}
    AllowUsers vagrant ansusenko
  \end{minted}
  \centering
  \includegraphics[width=\textwidth]{../images/5.png}
  \captionof{figure}{Восстановление доступа на сервер пользователю ansusenko.}
\end{frame}

\begin{frame}[fragile]
\frametitle{Настройка дополнительных портов для удалённого доступа по SSH}
В файле конфигурации \texttt{sshd\_config} добавили строки:
  \begin{minted}{bash}
    Port 22
    Port 2022
  \end{minted}
  \centering
  \includegraphics[width=\textwidth]{../images/6.png}
  \captionof{figure}{Проверка расширенного статуса работы sshd.}
\end{frame}

\begin{frame}
\frametitle{Настройка дополнительных портов для удалённого доступа по SSH}
  \centering
  \includegraphics[width=\textwidth]{../images/7.png}
  \captionof{figure}{Настройка межсетевого экрана.}
\end{frame}

\begin{frame}
\frametitle{Настройка дополнительных портов для удалённого доступа по SSH}
  \centering
  \includegraphics[width=\textwidth]{../images/8.png}
  \captionof{figure}{Успешное подключение к серверу.}
\end{frame}

\begin{frame}
\frametitle{Настройка дополнительных портов для удалённого доступа по SSH}
  \centering
  \includegraphics[width=\textwidth]{../images/9.png}
  \captionof{figure}{Успешное подключение к серверу по порту 2022.}
\end{frame}

\begin{frame}
\frametitle{Настройка удалённого доступа по SSH по ключу}
  \centering
  \includegraphics[width=\textwidth]{../images/10.png}
  \captionof{figure}{Копирование открытого ключа на сервер.}
\end{frame}

\begin{frame}
\frametitle{Настройка удалённого доступа по SSH по ключу}
  \centering
  \includegraphics[width=\textwidth]{../images/11.png}
  \captionof{figure}{Успешное подключение к серверу с использованием SSH-ключа.}
\end{frame}

\begin{frame}
\frametitle{Организация туннелей SSH, перенаправление TCP-портов}
  \centering
  \includegraphics[width=\textwidth]{../images/12.png}
  \captionof{figure}{Перенаправление TCP-портов.}
\end{frame}

\begin{frame}
\frametitle{Запуск консольных приложений через SSH}
  \centering
  \includegraphics[width=\textwidth]{../images/13.png}
  \captionof{figure}{Просмотр имени узла сервера и списка файлов через ssh.}
\end{frame}

\begin{frame}
\frametitle{Запуск консольных приложений через SSH}
  \centering
  \includegraphics[width=\textwidth]{../images/14.png}
  \captionof{figure}{Просмотр почты на сервере через ssh.}
\end{frame}

\begin{frame}[fragile]
\frametitle{Запуск графических приложений через SSH (X11Forwarding)}
Разрешили отображать на локальном клиентском компьютере графические интерфейсы X11:
  \begin{minted}{bash}
    X11Forwarding yes
  \end{minted}
Запустили графическое приложение на сервере:
  \begin{minted}{bash}
    ssh -Y -v ansusenko@server.ansusenko.net firefox
  \end{minted}
  \centering
  \includegraphics[width=0.7\textwidth]{../images/15.png}
  \captionof{figure}{Просмотр графического приложения (firefox) через ssh.}
\end{frame}

\begin{frame}[fragile]
\frametitle{Внесение изменений в настройки внутреннего окружения виртуальной машины}
  \begin{minted}[breaklines]{bash}
    cd /vagrant/provision/server
    mkdir -p /vagrant/provision/server/ssh/etc/ssh
    cp -R /etc/ssh/sshd_config /vagrant/provision/server/ssh/etc/ssh/
    cd /vagrant/provision/server
    touch ssh.sh
    chmod +x ssh.sh
  \end{minted}
\end{frame}

\begin{frame}[fragile]
\frametitle{Внесение изменений в настройки внутреннего окружения виртуальной машины}
  \begin{minted}[breaklines]{bash}
    #!/bin/bash
    echo "Provisioning script $0"
    echo "Copy configuration files"
    cp -R /vagrant/provision/server/ssh/etc/* /etc
    restorecon -vR /etc
    echo "Configure firewall"
    firewall-cmd --add-port=2022/tcp
    firewall-cmd --add-port=2022/tcp --permanent
    echo "Tuning SELinux"
    semanage port -a -t ssh_port_t -p tcp 2022
    echo "Restart sshd service"
    systemctl restart sshd
  \end{minted}
\end{frame}

\begin{frame}[fragile]
\frametitle{Внесение изменений в настройки внутреннего окружения виртуальной машины}
\begin{minted}{bash}
    server.vm.provision "server ssh",
    type: "shell",
    preserve_order: true,
    path: "provision/server/ssh.sh"
\end{minted}
\end{frame}

\begin{frame}[fragile]
\frametitle{Ответы на контрольные вопросы}
\begin{enumerate}
    \item \textbf{Как запретить удалённый доступ по SSH пользователю root и разрешить пользователю alice?}
    
    В файле \texttt{/etc/ssh/sshd\_config} нужно установить:
    \begin{minted}{bash}
PermitRootLogin no
AllowUsers alice
    \end{minted}
    Затем перезапустить службу: \texttt{systemctl restart sshd}
\end{enumerate}
\end{frame}

\begin{frame}[fragile]
\frametitle{Ответы на контрольные вопросы}
\begin{enumerate}
\setcounter{enumi}{1}
    \item \textbf{Как настроить удалённый доступ по SSH через несколько портов? Для чего это может потребоваться?}
    
    В файле \texttt{/etc/ssh/sshd\_config} добавить:
    \begin{minted}{bash}
Port 22
Port 2022
    \end{minted}
    Это может потребоваться для:
    \begin{itemize}
        \item Резервного подключения при блокировке основного порта
        \item Разделения доступа для разных групп пользователей
        \item Обхода ограничений межсетевого экрана
    \end{itemize}
\end{enumerate}
\end{frame}

\begin{frame}[fragile]
\frametitle{Ответы на контрольные вопросы}
\begin{enumerate}
\setcounter{enumi}{2}
    \item \textbf{Какие параметры используются для создания туннеля SSH в фоновом режиме?}
    
    Используются параметры:
    \begin{minted}{bash}
ssh -f -N -L локальный_порт:удаленный_хост:удаленный_порт пользователь@сервер
    \end{minted}
    где:
    \begin{itemize}
        \item \texttt{-f} - переход в фоновый режим
        \item \texttt{-N} - не выполнять удалённую команду
        \item \texttt{-L} - локальная переадресация портов
    \end{itemize}
\end{enumerate}
\end{frame}

\begin{frame}[fragile]
\frametitle{Ответы на контрольные вопросы}
\begin{enumerate}
\setcounter{enumi}{3}
    \item \textbf{Как настроить локальную переадресацию с локального порта 5555 на порт 80 сервера server2.example.com?}
    
    \begin{minted}{bash}
ssh -L 5555:server2.example.com:80 пользователь@промежуточный_сервер
    \end{minted}
    После этого обращение к \texttt{localhost:5555} будет перенаправляться на \texttt{server2.example.com:80} через промежуточный сервер.
\end{enumerate}
\end{frame}

\begin{frame}[fragile]
\frametitle{Ответы на контрольные вопросы}
\begin{enumerate}
\setcounter{enumi}{4}
    \item \textbf{Как настроить SELinux для работы SSH с портом 2022?}
    
    \begin{minted}{bash}
semanage port -a -t ssh_port_t -p tcp 2022
    \end{minted}
    Эта команда добавляет порт 2022 в список разрешённых портов для SSH в SELinux.
\end{enumerate}
\end{frame}

\begin{frame}[fragile]
\frametitle{Ответы на контрольные вопросы}
\begin{enumerate}
\setcounter{enumi}{5}
    \item \textbf{Как настроить межсетевой экран для разрешения SSH через порт 2022?}
    
    Для firewalld:
    \begin{minted}{bash}
firewall-cmd --add-port=2022/tcp
firewall-cmd --add-port=2022/tcp --permanent
    \end{minted}
    Для iptables:
    \begin{minted}{bash}
iptables -A INPUT -p tcp --dport 2022 -j ACCEPT
service iptables save
    \end{minted}
\end{enumerate}
\end{frame}

\begin{frame}
\frametitle{Выводы}
\begin{itemize}
    \item В результате выполнения лабораторной работы приобрели практические навыки по настройке удалённого доступа к серверу с помощью SSH.
    \item Освоили методы ограничения доступа пользователей и настройки дополнительных портов.
    \item Изучили организацию туннелей SSH и настройку аутентификации по ключам.
\end{itemize}
\end{frame}
\end{document}
