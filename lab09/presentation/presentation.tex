\input{preamble.tex}

\title{Лабораторная работа № 9. \\ Настройка POP3/IMAP сервера}
\author{Сущенко Алина \\ НПИбд-01-23}

\institute{Российский университет дружбы народов имени Патриса Лумумбы}
\date{2024}

\begin{document}

\frame{\titlepage}

\begin{frame}
\frametitle{Цель работы}
\begin{itemize}
    \item Приобретение практических навыков по установке и простейшему конфигурированию POP3/IMAP-сервера.
\end{itemize}
\end{frame}

\begin{frame}[containsverbatim]
\frametitle{Установка Dovecot}
Установили необходимые для работы пакеты:
\begin{minted}{bash}
sudo -i
dnf -y install dovecot telnet
\end{minted}
\end{frame}

\begin{frame}
\frametitle{Настройка протоколов Dovecot}
    \centering
    \includegraphics[width=\textwidth]{../images/01.png}
    \captionof{figure}{Разрешенные почтовые протоколы в конфигурации Dovecot.}
\end{frame}

\begin{frame}
\frametitle{Настройка аутентификации}
    \centering
    \includegraphics[width=\textwidth]{../images/02.png}
    \captionof{figure}{Указание метода аутентификации \texttt{plain}.}
\end{frame}

\begin{frame}
\frametitle{Настройка системной аутентификации}
    \centering
    \includegraphics[width=\textwidth]{../images/03.png}
    \captionof{figure}{Использование файла \texttt{passwd} для поиска паролей.}
\end{frame}

\begin{frame}
\frametitle{Настройка системной аутентификации}
    \centering
    \includegraphics[width=\textwidth]{../images/04.png}
    \captionof{figure}{Использование \texttt{pam} пользователей.}
\end{frame}

\begin{frame}
\frametitle{Настройка расположения почты}
    \centering
    \includegraphics[width=\textwidth]{../images/05.png}
    \captionof{figure}{Настройка месторасположения почтовых ящиков пользователей.}
\end{frame}

\begin{frame}
\frametitle{Настройка межсетевого экрана}
    \centering
    \includegraphics[width=\textwidth]{../images/06.png}
    \captionof{figure}{Настройка межсетевого экрана для работы служб протоколов POP3 и IMAP.}
\end{frame}

\begin{frame}
\frametitle{Запуск служб}
    \centering
    \includegraphics[width=\textwidth]{../images/07.png}
    \captionof{figure}{Перезапуск Postfix и Dovecot.}
\end{frame}

\begin{frame}
\frametitle{Проверка работы Dovecot}
    \centering
    \includegraphics[width=\textwidth]{../images/08.png}
    \captionof{figure}{Запуск мониторинга работы почтовой службы.}
\end{frame}

\begin{frame}
\frametitle{Просмотр почты на сервере}
    \centering
    \includegraphics[width=\textwidth]{../images/09.png}
    \captionof{figure}{Просмотр имеющейся почты на сервере.}
\end{frame}

\begin{frame}
\frametitle{Настройка почтового клиента Alpine}
    \centering
    \includegraphics[width=\textwidth]{../images/10.png}
    \captionof{figure}{Настройка входящих сообщений.}
\end{frame}

\begin{frame}
\frametitle{Отправка тестового письма}
    \centering
    \includegraphics[width=\textwidth]{../images/11.png}
    \captionof{figure}{Окно отправки письма.}
\end{frame}

\begin{frame}
\frametitle{Просмотр письма через Telnet}
    \centering
    \includegraphics[width=\textwidth]{../images/12.png}
    \captionof{figure}{Окно просмотра полученного письма через Telnet.}
\end{frame}

\begin{frame}[containsverbatim]
\frametitle{Автоматизация настройки}
Копирование конфигурационных файлов для provision:
\begin{minted}{bash}
cd /vagrant/provision/server
mkdir -p /vagrant/provision/server/mail/etc/dovecot/conf.d
cp -R /etc/dovecot/dovecot.conf /vagrant/provision/server/mail/etc/dovecot/
cp -R /etc/dovecot/conf.d/10-auth.conf /vagrant/provision/server/mail/etc/dovecot/conf.d/
cp -R /etc/dovecot/conf.d/auth-system.conf.ext /vagrant/provision/server/mail/etc/dovecot/conf.d/
cp -R /etc/dovecot/conf.d/10-mail.conf /vagrant/provision/server/mail/etc/dovecot/conf.d/
\end{minted}
\end{frame}

\begin{frame}[containsverbatim]
\frametitle{Скрипт автоматической настройки mail.sh}
\begin{minted}{bash}
#!/bin/bash
echo "Provisioning script $0"
echo "Install needed packages"
dnf -y install postfix
dnf -y install dovecot
dnf -y install telnet
echo "Copy configuration files"
cp -R /vagrant/provision/server/mail/etc/* /etc
chown -R root:root /etc/postfix
restorecon -vR /etc
\end{minted}
\end{frame}

\begin{frame}[containsverbatim]
\frametitle{Скрипт автоматической настройки mail.sh (продолжение)}
\begin{minted}{bash}
echo "Configure firewall"
firewall-cmd --add-service smtp --permanent
firewall-cmd --add-service pop3 --permanent
firewall-cmd --add-service pop3s --permanent
firewall-cmd --add-service imap --permanent
firewall-cmd --add-service imaps --permanent
firewall-cmd --add-service smtp-submission --permanent
firewall-cmd --reload
echo "Start postfix service"
systemctl enable postfix
systemctl start postfix
\end{minted}
\end{frame}

\begin{frame}[containsverbatim]
\frametitle{Скрипт автоматической настройки mail.sh (продолжение)}
\begin{minted}{bash}
echo "Configure postfix"
postconf -e 'mydomain = user.net'
postconf -e 'myorigin = $mydomain'
postconf -e 'inet_protocols = ipv4'
postconf -e 'inet_interfaces = all'
postconf -e 'mydestination = $myhostname, localhost.$mydomain, localhost, $mydomain'
echo "Configure postfix for dovecot"
postconf -e 'home_mailbox = Maildir/'
echo "Configure postfix for auth"
postconf -e 'smtpd_sasl_type = dovecot'
postconf -e 'smtpd_sasl_path = private/auth'
\end{minted}
\end{frame}

\begin{frame}[containsverbatim]
\frametitle{Скрипт автоматической настройки mail.sh (окончание)}
\begin{minted}{bash}
postconf -e 'smtpd_recipient_restrictions = reject_unknown_recipient_domain, permit_mynetworks, reject_non_fqdn_recipient, reject_unauth_destination, reject_unverified_recipient, permit'
postconf -e 'mynetworks = 127.0.0.0/8'
echo "Configure postfix for SMTP over TLS"
cp /etc/pki/dovecot/certs/dovecot.pem /etc/pki/tls/certs
cp /etc/pki/dovecot/private/dovecot.pem /etc/pki/tls/private
postconf -e 'smtpd_tls_cert_file=/etc/pki/tls/certs/dovecot.pem'
postconf -e 'smtpd_tls_key_file=/etc/pki/tls/private/dovecot.pem'
postconf -e 'smtpd_tls_session_cache_database = btree:/var/lib/postfix/smtpd_scache'
postconf -e 'smtpd_tls_security_level = may'
postconf -e 'smtp_tls_security_level = may'
postfix set-permissions
restorecon -vR /etc
systemctl stop postfix
systemctl start postfix
systemctl restart dovecot
\end{minted}
\end{frame}

\begin{frame}
\frametitle{Выводы}
\begin{itemize}
    \item В результате выполнения лабораторной работы приобрели практические навыки по установке и простейшему конфигурированию POP3/IMAP-сервера.
\end{itemize}
\end{frame}

\begin{frame}
\frametitle{Контрольные вопросы}

\begin{block}{1. За что отвечает протокол SMTP?}
SMTP (Simple Mail Transfer Protocol) отвечает за отправку и пересылку электронной почты между серверами. Он используется для передачи сообщений от отправителя к получателю через промежуточные серверы.
\end{block}

\begin{block}{2. За что отвечает протокол IMAP?}
IMAP (Internet Message Access Protocol) позволяет получать доступ к электронной почте с нескольких клиентских устройств, синхронизируя состояние писем между всеми устройствами. Письма хранятся на сервере.
\end{block}

\end{frame}

\begin{frame}
\frametitle{Контрольные вопросы}

\begin{block}{3. За что отвечает протокол POP3?}
POP3 (Post Office Protocol version 3) используется для загрузки почты с сервера на локальное устройство. Обычно письма после загрузки удаляются с сервера, что делает их доступными только на одном устройстве.
\end{block}

\begin{block}{4. В чём назначение Dovecot?}
Dovecot - это открытый IMAP и POP3 сервер для Linux/UNIX-систем. Он обеспечивает безопасный доступ пользователей к их почтовым ящикам, поддерживает различные методы аутентификации и обеспечивает надежное хранение почты.
\end{block}

\end{frame}

\begin{frame}
\frametitle{Контрольные вопросы}

\begin{block}{5. В каких файлах обычно находятся настройки работы Dovecot?}
Основные конфигурационные файлы Dovecot:
\begin{itemize}
\item \texttt{/etc/dovecot/dovecot.conf} - основной конфигурационный файл
\item \texttt{/etc/dovecot/conf.d/10-auth.conf} - настройки аутентификации
\item \texttt{/etc/dovecot/conf.d/10-mail.conf} - настройки почтовых ящиков
\item \texttt{/etc/dovecot/conf.d/10-ssl.conf} - настройки SSL/TLS
\item \texttt{/etc/dovecot/conf.d/15-lda.conf} - настройки агента доставки
\item \texttt{/etc/dovecot/conf.d/auth-system.conf.ext} - системная аутентификация
\end{itemize}
\end{block}

\end{frame}

\begin{frame}
\frametitle{Контрольные вопросы}

\begin{block}{6. В чём назначение Postfix?}
Postfix - это почтовый агент (MTA - Mail Transfer Agent), который отвечает за маршрутизацию и доставку электронной почты. Он принимает почту от клиентов и передает ее другим серверам или локальным почтовым ящикам.
\end{block}

\begin{block}{7. Какие методы аутентификации пользователей можно использовать в Dovecot?}
\begin{itemize}
\item \texttt{plain} - обычная текстовая аутентификация
\item \texttt{login} - устаревший метод для совместимости
\item \texttt{cram-md5} - challenge-response аутентификация
\item \texttt{digest-md5} - более безопасный метод
\item \texttt{gssapi} - для Kerberos
\item \texttt{oauth2} - OAuth 2.0 аутентификация
\end{itemize}
\end{block}

\end{frame}

\begin{frame}[containsverbatim]
\frametitle{Контрольные вопросы}

\begin{block}{8. Приведите пример заголовка письма с пояснениями его полей.}
\begin{minted}{text}
From: user@example.com          # Отправитель
To: recipient@domain.com        # Получатель  
Subject: Test Message           # Тема письма
Date: Mon, 15 Jan 2024 10:30:00 +0300  # Дата отправки
Message-ID: <123456@example.com> # Уникальный идентификатор
MIME-Version: 1.0              # Версия MIME
Content-Type: text/plain; charset=utf-8  # Тип содержимого
\end{minted}
\end{block}

\end{frame}

\begin{frame}[containsverbatim]
\frametitle{Контрольные вопросы}

\begin{block}{9. Приведите примеры использования команд для работы с почтовыми протоколами через терминал}
\begin{minted}{bash}
# Подключение к POP3 серверу
telnet mail.example.com 110
USER username
PASS password
LIST        # список писем
RETR 1      # получить письмо №1
QUIT

# Подключение к IMAP серверу  
telnet mail.example.com 143
a1 LOGIN username password
a2 LIST "" "*"     # список папок
a3 SELECT INBOX    # выбрать папку входящих
a4 FETCH 1 BODY[TEXT]  # получить текст письма №1
a5 LOGOUT
\end{minted}
\end{block}

\end{frame}

\begin{frame}[containsverbatim]
\frametitle{Контрольные вопросы}

\begin{block}{10. Приведите примеры с пояснениями по работе с doveadm}
\begin{minted}{bash}
# Просмотр информации о пользователе
doveadm user username

# Поиск писем по критериям
doveadm search -u username mailbox INBOX

# Получение содержимого письма
doveadm fetch -u username "body peek.text"

# Экспорт почтового ящика в формат mbox
doveadm backup -u username -f mbox

# Статистика работы сервера
doveadm stats

# Перестроение индексов для восстановления
doveadm index -u username INBOX
\end{minted}
\end{block}

\end{frame}

\end{document}
