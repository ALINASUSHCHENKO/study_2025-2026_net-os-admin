\input{preamble.tex}

\title{Лабораторная работа № 8. \\ Настройка SMTP-сервера}
\author{Сущенко Алина \\ НПИбд-01-23}

\institute{Российский университет дружбы народов имени Патриса Лумумбы}
\date{2025}

\begin{document}

\frame{\titlepage}

\begin{frame}
\frametitle{Цель работы}
\begin{itemize}
    \item Приобретение практических навыков по установке и конфигурированию SMTP-сервера.
\end{itemize}
\end{frame}

\begin{frame}[containsverbatim]
\frametitle{Установка Postfix}
Установили необходимые для работы пакеты:
\begin{minted}{bash}
dnf -y install postfix
dnf -y install s-nail
\end{minted}
    \centering
    \includegraphics[width=\textwidth]{../images/01.png}
    \captionof{figure}{Конфигурирование межсетевого экрана.}
\end{frame}

\begin{frame}
\frametitle{Установка Postfix}
    \centering
    \includegraphics[width=\textwidth]{../images/02.png}
    \captionof{figure}{Восстановление контекста безопасности SELinux.}
\end{frame}

\begin{frame}
\frametitle{Установка Postfix}
    \centering
    \includegraphics[width=\textwidth]{../images/03.png}
    \captionof{figure}{Запуск службы Postfix.}
\end{frame}

\begin{frame}
\frametitle{Изменение параметров Postfix с помощью postconf}
    \centering
    \includegraphics[width=\textwidth]{../images/04.png}
    \captionof{figure}{Просмотр текущих настроек Postfix.}
\end{frame}

\begin{frame}
\frametitle{Изменение параметров Postfix с помощью postconf}
    \centering
    \includegraphics[width=\textwidth]{../images/05.png}
    \captionof{figure}{Замена значений параметров Postfix.}
\end{frame}

\begin{frame}
\frametitle{Изменение параметров Postfix с помощью postconf}
    \centering
    \includegraphics[width=\textwidth]{../images/06.png}
    \captionof{figure}{Проверка корректности содержания конфигурационного файла \texttt{main.cf}.}
\end{frame}

\begin{frame}
\frametitle{Изменение параметров Postfix с помощью postconf}
    \centering
    \includegraphics[width=\textwidth]{../images/07.png}
    \captionof{figure}{Перезагрузка службы Postfix.}
\end{frame}

\begin{frame}
\frametitle{Изменение параметров Postfix с помощью postconf}
    \centering
    \includegraphics[width=\textwidth]{../images/08.png}
    \captionof{figure}{Просмотр всех параметров Postfix с отличиями от начальных.}
\end{frame}

\begin{frame}
\frametitle{Изменение параметров Postfix с помощью postconf}
    \centering
    \includegraphics[width=\textwidth]{../images/09.png}
    \captionof{figure}{Настройка Postfix на работу только с протоколом IPv4.}
\end{frame}

\begin{frame}
\frametitle{Изменение параметров Postfix с помощью postconf}
    \centering
    \includegraphics[width=\textwidth]{../images/10.png}
    \captionof{figure}{Перезагрузка службы Postfix.}
\end{frame}

\begin{frame}[containsverbatim]
\frametitle{Проверка работы Postfix}
\begin{minted}{bash}
echo .| mail -s test1 ansusenko@server.ansusenko.net
\end{minted}
    \centering
    \includegraphics[width=\textwidth]{../images/11.png}
    \captionof{figure}{Просмотр логов почтовой службы при отправке сообщения.}
\end{frame}

\begin{frame}[containsverbatim]
\frametitle{Проверка работы Postfix}
Установили необходимые для работы пакеты:
\begin{minted}{bash}
dnf -y install postfix
dnf -y install s-nail
\end{minted}
    \centering
    \includegraphics[width=\textwidth]{../images/12.png}
    \captionof{figure}{Установка Postfix.}
\end{frame}

\begin{frame}
\frametitle{Проверка работы Postfix}
    \centering
    \includegraphics[width=\textwidth]{../images/13.png}
    \captionof{figure}{Настройка Postfix на работу только с протоколом IPv4 на клиенте.}
\end{frame}

\begin{frame}
\frametitle{Проверка работы Postfix}
    \centering
    \includegraphics[width=\textwidth]{../images/14.png}
    \captionof{figure}{Перезапуск службы Postfix.}
\end{frame}

\begin{frame}
\frametitle{Проверка работы Postfix}
    \centering
    \includegraphics[width=\textwidth]{../images/16.png}
    \captionof{figure}{Логи при попытке отправить сообщение с клиента.}
\end{frame}

\begin{frame}
\frametitle{Проверка работы Postfix}
    \centering
    \includegraphics[width=\textwidth]{../images/17.png}
    \captionof{figure}{Просмотр значений параметров сетевых интерфейсов.}
\end{frame}

\begin{frame}
\frametitle{Проверка работы Postfix}
    \centering
    \includegraphics[width=\textwidth]{../images/18.png}
    \captionof{figure}{Добавление адреса внутренней сети для разрешения пересылки сообщений между узлами сети.}
\end{frame}

\begin{frame}
\frametitle{Проверка работы Postfix}
    \centering
    \includegraphics[width=\textwidth]{../images/19.png}
    \captionof{figure}{Перезагрузка службы Postfix.}
\end{frame}

\begin{frame}
\frametitle{Проверка работы Postfix}
    \centering
    \includegraphics[width=\textwidth]{../images/20.png}
    \captionof{figure}{Логи при повторной отправке сообщения с клиента.}
\end{frame}

\begin{frame}
\frametitle{Конфигурация Postfix для домена}
    \centering
    \includegraphics[width=\textwidth]{../images/20.png}
    \captionof{figure}{Логи при отправке сообщения на доменный адрес.}
\end{frame}

\begin{frame}
\frametitle{Конфигурация Postfix для домена}
    \centering
    \includegraphics[width=\textwidth]{../images/21.png}
    \captionof{figure}{Просмотр сообщений в очереди на отправление.}
\end{frame}

\begin{frame}
\frametitle{Конфигурация Postfix для домена}
    \centering
    \includegraphics[width=\textwidth]{../images/22.png}
    \captionof{figure}{Обновление конфигурации Postfix.}
\end{frame}

\begin{frame}
\frametitle{Конфигурация Postfix для домена}
    \centering
    \includegraphics[width=\textwidth]{../images/23.png}
    \captionof{figure}{Логи при отправке сообщения.}
\end{frame}

\begin{frame}
\frametitle{Внесение изменений в настройки внутреннего окружения виртуальной машины}
    \centering
    \includegraphics[width=\textwidth]{../images/24.png}
    \captionof{figure}{Создание каталога для настроек внутреннего окружения.}
\end{frame}

\begin{frame}[containsverbatim]
\frametitle{Внесение изменений в настройки внутреннего окружения виртуальной машины}
\begin{minted}{bash}
#!/bin/bash
echo "Provisioning script $0"
echo "Install needed packages"
dnf -y install postfix
dnf -y install s-nail
echo "Copy configuration files"
cp -R /vagrant/provision/server/mail/etc/* /etc
echo "Configure firewall"
firewall-cmd --add-service=smtp --permanent
firewall-cmd --reload
restorecon -vR /etc
echo "Start postfix service"
systemctl enable postfix
systemctl start postfix
\end{minted}
\end{frame}

\begin{frame}[containsverbatim]
\frametitle{Внесение изменений в настройки внутреннего окружения виртуальной машины}
\begin{minted}{bash}
echo "Configure postfix"
postconf -e 'mydomain = dastarikov.net'
postconf -e 'myorigin = $mydomain'
postconf -e 'inet_protocols = ipv4'
postconf -e 'inet_interfaces = all'
postconf -e 'mydestination = $myhostname, localhost.$mydomain, localhost, $mydomain'
postconf -e 'mynetworks = 127.0.0.0/8, 192.168.0.0/16'
postfix set-permissions
restorecon -vR /etc
systemctl stop postfix
systemctl start postfix
\end{minted}
\end{frame}

\begin{frame}
\frametitle{Выводы}
\begin{itemize}
    \item В результате выполнения лабораторной работы приобрели практические навыки по установке и конфигурированию SMTP-сервера.
\end{itemize}
\end{frame}

\begin{frame}
\frametitle{Контрольные вопросы}

\begin{block}{1. В каком каталоге и в каком файле следует смотреть конфигурацию Postfix?}
Основная конфигурация Postfix находится в каталоге \texttt{/etc/postfix/}:
\begin{itemize}
\item \texttt{/etc/postfix/main.cf} — основной файл конфигурации с параметрами сервера
\item \texttt{/etc/postfix/master.cf} — конфигурация процессов и служб
\item \texttt{/etc/aliases} — файл псевдонимов почтовых ящиков
\item \texttt{/etc/postfix/access} — контроль доступа к серверу
\end{itemize}
\end{block}

\begin{block}{2. Каким образом можно проверить корректность синтаксиса в конфигурационном файле Postfix?}
\begin{itemize}
\item \texttt{postfix check} — проверка синтаксиса всех конфигурационных файлов
\item \texttt{postconf -n} — просмотр всех измененных параметров
\item \texttt{postconf <параметр>} — проверка конкретного параметра
\item \texttt{systemctl status postfix} — проверка статуса службы
\item Просмотр логов: \texttt{tail -f /var/log/maillog}
\end{itemize}
\end{block}

\end{frame}

\begin{frame}
\frametitle{Контрольные вопросы}

\begin{block}{3. В каких параметрах конфигурации Postfix требуется внести изменения для настройки отправки писем на доменные адреса?}
\begin{itemize}
\item \texttt{mydomain = example.com} — установка доменного имени
\item \texttt{myorigin = \$mydomain} — домен для исходящей почты
\item \texttt{mydestination = \$myhostname, localhost.\$mydomain, localhost, \$mydomain} — домены для локальной доставки
\item \texttt{inet\_interfaces = all} — прослушивание всех интерфейсов
\item \texttt{mynetworks = 127.0.0.0/8, 192.168.0.0/16} — разрешенные сети для релейинга
\item \texttt{relay\_domains = \$mydestination} — домены для релейинга
\end{itemize}
\end{block}

\end{frame}

\begin{frame}[containsverbatim]
\frametitle{Контрольные вопросы}

\begin{block}{4. Примеры работы с утилитой mail}
\begin{minted}{bash}
# Отправка письма с темой и текстом
echo "Текст письма" | mail -s "Тема письма" user@example.com

# Отправка письма с содержимым файла
mail -s "Тема" user@example.com < file.txt

# Просмотр списка писем
mail

# Просмотр конкретного письма (внутри mail)
print 1

# Удаление письма (внутри mail)
delete 1

# Выход из mail с сохранением писем
quit

# Выход с удалением прочитанных писем
exit
\end{minted}
\end{block}

\end{frame}

\begin{frame}[containsverbatim]
\frametitle{Контрольные вопросы}

\begin{block}{5. Примеры работы с утилитой postqueue}
\begin{minted}{bash}
# Просмотр всей очереди сообщений
postqueue -p

# Подсчет числа сообщений в очереди
postqueue -p | grep -c "^[A-F0-9]"

# Альтернативный способ подсчета
mailq | wc -l

# Принудительная отправка всех сообщений в очереди
postqueue -f

# Принудительная отправка сообщений для определенного домена
postqueue -s example.com

# Удаление всех сообщений из очереди
postsuper -d ALL

# Удаление конкретного сообщения по ID
postsuper -d A1B2C3D4E5

# Удаление всех отложенных сообщений
postsuper -d ALL deferred
\end{minted}
\end{block}

\end{frame}

\end{document}
