\input{preamble.tex}

\title{Лабораторная работа № 14. \ Настройка файловых служб Samba}
\author{Сущенко Алина \ НПИбд-01-23}
\institute{Российский университет дружбы народов имени Патриса Лумумбы}
\date{2025}

\begin{document}

\frame{\titlepage}

\begin{frame}
\frametitle{Цель работы}
\begin{itemize}
\item Приобретение навыков настройки доступа групп пользователей к общим ресурсам по протоколу SMB.
\end{itemize}
\end{frame}

\begin{frame}
\frametitle{Настройка сервера Samba}
\includegraphics[width=0.8\textwidth]{../images/01.png}
\captionof{figure}{Создание группы sambagroup и добавления в нее пользователя ansusenko.}
\end{frame}

\begin{frame}
\frametitle{Настройка сервера Samba}
\includegraphics[width=0.8\textwidth]{../images/02.png}
\captionof{figure}{Изменение конфигурации samba на сервере.}
\end{frame}

\begin{frame}
\frametitle{Настройка сервера Samba}
\includegraphics[width=0.8\textwidth]{../images/03.png}
\captionof{figure}{Проверка правильности синтаксиса файла samba.conf.}
\end{frame}

\begin{frame}
\frametitle{Настройка сервера Samba}
\includegraphics[width=0.8\textwidth]{../images/04.png}
\captionof{figure}{Включение демона Samba.}
\end{frame}

\begin{frame}
\frametitle{Настройка сервера Samba}
\includegraphics[width=0.8\textwidth]{../images/05.png}
\captionof{figure}{Проверка наличия общего доступа к серверу.}
\end{frame}

\begin{frame}
\frametitle{Настройка сервера Samba}
\includegraphics[width=0.8\textwidth]{../images/06.png}
\captionof{figure}{Просмотр конфигурации межсетевого экрана Samba.}
\end{frame}

\begin{frame}
\frametitle{Настройка сервера Samba}
\includegraphics[width=0.8\textwidth]{../images/07.png}
\captionof{figure}{Настройка межсетевого экрана.}
\end{frame}

\begin{frame}
\frametitle{Настройка сервера Samba}
\includegraphics[width=0.8\textwidth]{../images/08.png}
\captionof{figure}{Настройка прав доступа для каталога с разделяемым ресурсом sambashare.}
\end{frame}

\begin{frame}
\frametitle{Настройка сервера Samba}
\includegraphics[width=0.8\textwidth]{../images/09.png}
\captionof{figure}{Настройка SELinux для каталога с разделяемым ресурсом smbshare.}
\end{frame}

\begin{frame}
\frametitle{Настройка сервера Samba}
\includegraphics[width=0.8\textwidth]{../images/10.png}
\captionof{figure}{Настройка разрешений через флаги SELinux.}
\end{frame}

\begin{frame}
\frametitle{Настройка сервера Samba}
\includegraphics[width=0.8\textwidth]{../images/11.png}
\captionof{figure}{Просмотр групп пользователя ansusenko.}
\end{frame}

\begin{frame}
\frametitle{Настройка сервера Samba}
\includegraphics[width=0.8\textwidth]{../images/12.png}
\captionof{figure}{Создание файла на разделяемой ресурсе.}
\end{frame}

\begin{frame}
\frametitle{Настройка сервера Samba}
\includegraphics[width=0.8\textwidth]{../images/13.png}
\captionof{figure}{Добавление ansusenko в базу пользователей Samba.}
\end{frame}

\begin{frame}
\frametitle{Монтирование файловой системы Samba на клиенте}
\includegraphics[width=0.8\textwidth]{../images/14.png}
\captionof{figure}{Просмотр конфигурации межсетевого экрана для клиента Samba.}
\end{frame}

\begin{frame}
\frametitle{Монтирование файловой системы Samba на клиенте}
\includegraphics[width=0.8\textwidth]{../images/15.png}
\captionof{figure}{Настройка межсетевого экрана клиента.}
\end{frame}

\begin{frame}
\frametitle{Монтирование файловой системы Samba на клиенте}
\includegraphics[width=0.8\textwidth]{../images/16.png}
\captionof{figure}{Создание аналогичной группы на клиенте и добавления в нее ansusenko.}
\end{frame}

\begin{frame}
\frametitle{Монтирование файловой системы Samba на клиенте}
\includegraphics[width=0.8\textwidth]{../images/17.png}
\captionof{figure}{Изменение конфигурации samba на клиенте.}
\end{frame}

\begin{frame}
\frametitle{Монтирование файловой системы Samba на клиенте}
\includegraphics[width=0.8\textwidth]{../images/18.png}
\captionof{figure}{Подключение к серверу под пользователем root.}
\end{frame}

\begin{frame}
\frametitle{Монтирование файловой системы Samba на клиенте}
\includegraphics[width=0.8\textwidth]{../images/19.png}
\captionof{figure}{Подключение к серверу под пользователем ansusenko.}
\end{frame}

\begin{frame}
\frametitle{Монтирование файловой системы Samba на клиенte}
\includegraphics[width=0.8\textwidth]{../images/20.png}
\captionof{figure}{Монтирование разделяемого ресурса на клиенте.}
\end{frame}

\begin{frame}
\frametitle{Монтирование файловой системы Samba на клиенте}
\includegraphics[width=0.8\textwidth]{../images/21.png}
\captionof{figure}{Проверка возможности создавать файлы на разделяемом ресурсе.}
\end{frame}

\begin{frame}
\frametitle{Монтирование файловой системы Samba на клиенте}
\includegraphics[width=0.8\textwidth]{../images/22.png}
\captionof{figure}{Отмонтирование каталога samba и создание файла smbusers.}
\end{frame}

\begin{frame}
\frametitle{Монтирование файловой системы Samba на клиенте}
\includegraphics[width=0.8\textwidth]{../images/23.png}
\captionof{figure}{Изменение файла /etc/fstab.}
\end{frame}

\begin{frame}
\frametitle{Монтирование файловой системы Samba на клиенте}
\includegraphics[width=0.8\textwidth]{../images/24.png}
\captionof{figure}{Настройка автоматического монтирования каталога для разделяемых ресурсов.}
\end{frame}

\begin{frame}
\frametitle{Ответы на контрольные вопросы}
\begin{enumerate}
\item Минимальная конфигурация для smb.conf для создания общего ресурса, который предоставляет доступ к каталогу /data:
\begin{verbatim}
[data]
path = /data
browseable = yes
read only = no
guest ok = no
\end{verbatim}
\end{enumerate}
\end{frame}

\begin{frame}
\frametitle{Ответы на контрольные вопросы}
\begin{enumerate}
\setcounter{enumi}{1}
\item Настроить общий ресурс с доступом на запись всем пользователям, имеющим права на запись в файловой системе Linux:
\begin{verbatim}
[shared]
path = /srv/samba/shared
browseable = yes
read only = no
guest ok = no
writeable = yes
\end{verbatim}
\end{enumerate}
\end{frame}

\begin{frame}
\frametitle{Ответы на контрольные вопросы}
\begin{enumerate}
\setcounter{enumi}{2}
\item Ограничить доступ на запись к ресурсу только членам определённой группы:
\begin{verbatim}
[restricted]
path = /srv/samba/restricted
browseable = yes
read only = no
write list = @sambagroup
valid users = @sambagroup
\end{verbatim}
\end{enumerate}
\end{frame}

\begin{frame}
\frametitle{Ответы на контрольные вопросы}
\begin{enumerate}
\setcounter{enumi}{3}
\item Переключатель SELinux для доступа к домашним каталогам через SMB:
\begin{verbatim}
setsebool -P samba_enable_home_dirs on
\end{verbatim}
\end{enumerate}
\end{frame}

\begin{frame}
\frametitle{Ответы на контрольные вопросы}
\begin{enumerate}
\setcounter{enumi}{4}
\item Ограничить доступ к ресурсу только узлам из сети 192.168.10.0/24:
\begin{verbatim}
[restricted]
path = /srv/samba/restricted
hosts allow = 192.168.10.0/24
hosts deny = 0.0.0.0/0
\end{verbatim}
\end{enumerate}
\end{frame}

\begin{frame}
\frametitle{Ответы на контрольные вопросы}
\begin{enumerate}
\setcounter{enumi}{5}
\item Команда для отображения списка всех пользователей Samba на сервере:
\begin{verbatim}
pdbedit -L
\end{verbatim}
\end{enumerate}
\end{frame}

\begin{frame}
\frametitle{Ответы на контрольные вопросы}
\begin{enumerate}
\setcounter{enumi}{6}
\item Для доступа к многопользовательскому ресурсу пользователю необходимо:
\begin{itemize}
\item Иметь учётную запись на сервере Samba
\item Быть добавленным в базу пользователей Samba (smbpasswd -a)
\item Принадлежать к группе, которой разрешён доступ к ресурсу
\end{itemize}
\end{enumerate}
\end{frame}

\begin{frame}
\frametitle{Ответы на контрольные вопросы}
\begin{enumerate}
\setcounter{enumi}{7}
\item Установить общий ресурс Samba как многопользовательскую учётную запись с alice в качестве минимальной учётной записи:
\begin{verbatim}
[multiuser]
path = /srv/samba/multiuser
browseable = yes
read only = no
guest ok = no
force user = alice
\end{verbatim}
\end{enumerate}
\end{frame}

\begin{frame}
\frametitle{Ответы на контрольные вопросы}
\begin{enumerate}
\setcounter{enumi}{8}
\item Запретить пользователям просматривать учётные данные монтирования Samba в /etc/fstab:
\begin{itemize}
\item Использовать файл credentials с ограничением прав доступа (600)
\item В /etc/fstab указать: \texttt{credentials=/etc/samba/credentials}
\end{itemize}
\end{enumerate}
\end{frame}

\begin{frame}
\frametitle{Ответы на контрольные вопросы}
\begin{enumerate}
\setcounter{enumi}{9}
\item Команда для перечисления всех экспортируемых ресурсов Samba на сервере:
\begin{verbatim}
smbclient -L server_name -U username
\end{verbatim}
\end{enumerate}
\end{frame}

\begin{frame}
\frametitle{Выводы}
\begin{itemize}
\item В результате лабораторной работы познакомились с настройкой доступа групп пользователей к общим ресурсам по протоколу SMB.
\item Освоили настройку сервера Samba, управление доступом пользователей и групп.
\item Изучили монтирование файловых систем Samba на клиенте и настройку автоматического монтирования.
\end{itemize}
\end{frame}
\end{document}
